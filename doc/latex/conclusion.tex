\section{Summary and Outlook}

We have demonstrated that instabilities in traffic modelled by the IDM can be suppressed by adapting the breaking behaviour of the drivers. In particular increasing the interaction exponent $\gamma$ beyond a critical value leads to a complete suppression of any perturbation.

We think this can be understood by realizing that, if the drivers slow down earlier, the \emph{extend} of a perturbation grows and the velocities are equalized again. To further test this hypothesis, additional simulations will need to be carried out. One could e.g. trace the acceleration of vehicles and examine them for evidence whether changing $\gamma$ does indeed have an effect on the vehicles' behaviour as predicted.

One could also think of \emph{empirically} measuring the value for $\gamma$ for real drivers. This seems like a difficult task however, as all the parameters in the IDM are tightly coupled. Still a measurement of $\gamma$ could be used as a test for the validity of the IDM, as in a situation where in reality instabilities can be observed, we should see a value for $\gamma$ that is in the `unstable' region of e.g. the phase diagram in Figure \ref{fig:phase_diagram}.