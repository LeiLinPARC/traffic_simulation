\section{Description of the Model}
\label{sec:model}
We have implemented the \emph{Intelligent Driver Model} (IDM). The IDM, first introduced by Treiber et al. \cite{treiber1999, treiber2000}, is deterministic and continuous (both in time and space). It describes the movement of each vehicle by a set of differential equations, giving the vehicle's acceleration as a function of its current velocity $v_\alpha$, the gap to the leading vehicle $s_\alpha$, etc. 

Formally the acceleration $\dot v_\alpha$ of vehicle $\alpha$ reads 
\begin{equation}
\dot v_\alpha = a\left(\underbrace{1-\left(\frac{v_\alpha}{v_0}\right)^\delta}_{\text{free road term}} - \underbrace{\left(\frac{s^*(v_\alpha, \Delta v_\alpha)}{s_\alpha}\right)^\gamma}_{\text{inteaction term}}\right).
\label{eq:IDM}
\end{equation}


This equation contains two parts, the \emph{free road term}, and the \emph{interaction term}. For a low vehicle density (i.e. $s_\alpha \rightarrow 0$) the above equation can be simplified to
\begin{equation}
\dot v_\alpha = a\left(1-\left(\frac{v_\alpha}{v_0}\right)^\delta\right).
\end{equation}
For any $\delta>0$ this will result a relaxation of the velocity $v_\alpha$ to the speed limit $v_0$. The value of the exponent $\delta$ is typically chosen in between 1 (exponential relaxation) and 6 (the limit $\delta\rightarrow \infty$ corresponds to a constant acceleration). Here $a$ denotes the maximum acceleration. Throughout literature the most commonly used value for the exponent is $\delta=4$. This is the value that has also been adopted in this work.

\subsection{The interaction term}
The \emph{interaction term} merits further attention. It is designed to limit the cars' velocity on a densely populated road. It acts as a `repulsive potential', which tries to enforce the \emph{desired gap} $s^*$ as the distance between two cars. The desired gap is calculated as
\begin{equation}
s^*(v_\alpha, \Delta v_\alpha) = s_0 + \max\left(v_\alpha T + \frac{v_\alpha \Delta v_\alpha}{2\sqrt{ab}},\;0\right).
\label{eq:desired_gap}
\end{equation}
It contains
\begin{itemize}
    \item a minimum distance $s_0$, which is respected even at stand still,
    \item a safety distance $v_\alpha T$ based on the drivers' desired time headway\footnote{The time headway $T$ is the time between the moment the first vehicle's bumper passes a stationary point, and the moment the next vehicle passes it.} $T$,
    \item a breaking term $\frac{v_\alpha \Delta v_\alpha}{2\sqrt{ab}}$, with the `comfortable' deceleration $b$.
\end{itemize}
The breaking term inhibits large closing speeds between vehicles. Indeed, for the traditional value $\gamma=2$, and large closing speeds, the interaction term essentially reduces to 
\begin{equation}
    -a \left(\frac{v \Delta v}{2 \sqrt{a b}s}\right)^2 = -\frac{b_k^2}{b},
\end{equation}
where $b_k$ denotes the \emph{kinematic deceleration}, i.e. the deceleration necessary to avoid a collision. Since drivers thus underestimate situations where $b_k\ll b$, while they break harder than necessary for $b_k\gg b$, it is easy to see that the resulting acceleration will tend towards $\dot v = -b$. \cite{treiber2000}

In the present work, particular focus has been paid precisely to  $\gamma \ne 2$. We will try to answer the question to what extend this parameter affects the stability of the flow. In the following we will demonstrate that this increase of $\gamma$, while leaving the other parameters fixed, will suppress any instabilities in the traffic flow.

\subsection{Boundary conditions}
As for the boundary conditions of the road, there are two obvious choices: either add and remove vehicles during the simulation run (as is done e.g. in \cite{treiber1999}), or make the road periodic, i.e. ring-like \cite{treiber2015}. It is this second variant that has been implemented for this project. It makes the implementation significantly simpler, as no care has to be taken to spawn the vehicles at the right position with an appropriate speed. On the other hand, it provides limited ability to change the amount of cars on the road at run-time.

On the subject of periodic boundary conditions, it is worth noting that some experimental data exists on precisely such a situation, namely cars driving on a circular track \cite{nakayama2009,tadaki2013}. Here a spontaneously emerging instability has been observed; we will see that our model can qualitatively reproduce these observations.


\subsection{Limits of the model}
The IDM does not model the finite reaction time of an actual human driver\footnote{The finite time step $\Delta t$ is \emph{not} to be mistaken for a reaction time. Refer to \cite{treiber2006} for an extensive explanation of the differences.}. Indeed, when a realistic reaction time was included, IDM-like simulations show very poor stability when compared to empiric human driving. This suggests that in reality drivers do look ahead further than only the nearest neighbour \cite{treiber2006}.

The road on which the vehicles travel is assumed to be single lane (no overtaking) and straight (constant speed limit). All drivers and vehicles are assumed to be identical.

\subsection{Summary of the parameters}
In table \ref{tab:param} the reader may find an overview of all the parameters in the model. The indicated values are those used in the presented simulations unless specified otherwise.
\begin{table}
    \begin{center}
        \begin{tabular}[pos]{p{0.5\textwidth} p{0.3\textwidth}}
            \toprule
            Parameter & Typical value\\
            \midrule
            global speed limit $v_0$ & \SI{15}{m/s} \\
            maximum acceleration $a$ & \SI{0.6}{m/s^2} \\
            maximum comfortable deceleration $b$ & \SI{1.1}{m/s^2} \\
            minimum time headway $T$ & \SI{1.5}{s} \\
            minimum jam distance $s_0$ & \SI{2}{m} \\
            free road acceleration exponent $\delta$ & 4 \\
            interaction exponent $\gamma$ & 2\\
            car size $l_\alpha$ & \SI{5}{m} \\
            number of cars $N$ & 50\\
            length of road $L$ & \SI{1}{km} \\
            time step $\Delta t$ & \SI{0.250}{s} \\
            end time (duration of simulation) $t_\mathrm{end}$ &  \SI{30000}{s}\\
            \bottomrule
        \end{tabular}
    \end{center}
    \caption{Overview of the parameters in the simulation.}
    \label{tab:param}
\end{table}


Note that the four last parameters are not directly included in the formulation of the IDM as in eq. (\ref{eq:IDM}). Instead they are related to the numeric implementation.
